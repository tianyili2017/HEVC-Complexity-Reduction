\documentclass[a4paper,11pt]{jctvcdoc}

\usepackage{geometry}[2010/02/12]

\usepackage{hyperref}
\hypersetup{colorlinks=true}
\usepackage{color,soul}

\usepackage[position=bottom]{subfig}
\captionsetup[subfloat]{position=top}
\usepackage{multirow}
\usepackage{dcolumn}
\newcolumntype{.}{D{.}{.}{-1}}
\usepackage{colortbl}
\usepackage{makecell}
\usepackage{longtable}
\usepackage{array}
\usepackage{algorithm2e}

\usepackage[strings]{underscore}
\usepackage{csquotes}
\MakeOuterQuote{"}
\EnableQuotes

\newcommand\None{}
\newcommand\NotSet{}
\makeatletter
\newcommand{\Option}[1]{\ifx\optOption\@empty\gdef\optOption{#1}\else\g@addto@macro\optOption{ \\ #1}\fi}
\newcommand{\ShortOption}[1]{\ifx\optShortOption\@empty\gdef\optShortOption{#1}\else\g@addto@macro\optShortOption{ \\ #1}\fi}
\newcommand{\Default}[1]{\ifx\optDefault\@empty\gdef\optDefault{#1}\else\g@addto@macro\optDefault{ \\ #1}\fi}
\newcommand{\clearOptions}{\gdef\optOption{}\gdef\optShortOption{}\gdef\optDefault{}}
\makeatother
\newenvironment{OptionTable}[1]{%
	\footnotesize
	\def\arraystretch{1.8}
	\clearOptions
	\begin{longtable}{l<{\makecell[tl]{\optOption}}%
	                  >{\texttt\bgroup}l<{\makecell[tl]{\optShortOption}\egroup}%
	                  c<{\makecell[tc]{\optDefault}}%
	                  >{\def\arraystretch{1.0}}p{0.5\textwidth}<{\clearOptions}}
	\caption{#1} \\
	\hspace*{12em}&&\hspace*{8em}&\kill
	\hline
	 \thead{Option} &
	 \egroup\thead{Shorthand}\bgroup &
	 \thead{Default} &
	 \thead{Description} \\
	\hline
	\endfirsthead
	\caption[]{#1 (Continued)} \\
	\hspace*{12em}&&\hspace*{8em}&\kill
	\hline
	 \thead{Option} &
	 \egroup\thead{Shorthand}\bgroup &
	 \thead{Default} &
	 \thead{Description} \\
	\hline
	\endhead
	 \multicolumn{4}{r}{Continued...}\\
	 \hline
	\endfoot
	 \hline
	\endlastfoot
}{%
	\hline
	\end{longtable}
}

\newenvironment{OptionTableNoShorthand}[2]{%
	\footnotesize
	\def\arraystretch{1.8}
	\clearOptions
	\begin{longtable}{l<{\makecell[tl]{\optOption}}%
	                  c<{\makecell[tc]{\optDefault}}%
	                  >{\def\arraystretch{1.0}}p{0.5\textwidth}<{\clearOptions}}
	\caption{#1} \label{#2} \\
	\hspace*{12em}&\hspace*{8em}&\kill
	\hline
	 \thead{Option} &
	 \thead{Default} &
	 \thead{Description} \\
	\hline
	\endfirsthead
	\caption[]{#1 (Continued)} \\
	\hspace*{12em}&\hspace*{8em}&\kill
	\hline
	 \thead{Option} &
	 \thead{Default} &
	 \thead{Description} \\
	\hline
	\endhead
	 \multicolumn{3}{r}{Continued...}\\
	 \hline
	\endfoot
	 \hline
	\endlastfoot
}{%
	\hline
	\end{longtable}
}

\newenvironment{SEIListTable}[1]{%
	\footnotesize
	\def\arraystretch{1.8}
	\clearOptions
	\begin{longtable}{c<{\makecell[tl]{\optOption}}%
	                  l<{\makecell[tc]{\optDefault}}%
	                  >{\def\arraystretch{1.0}}p{0.3\textwidth}<{\clearOptions}}
	\caption{#1} \\
	\hspace*{12em}&\hspace*{8em}&\kill
	\hline
	 \thead{SEI Number} &
	 \thead{SEI Name} &
	 \thead{Table number of encoder controls, if available} \\
	\hline
	\endfirsthead
	\caption[]{#1 (Continued)} \\
	\hspace*{12em}&\hspace*{8em}&\kill
	\hline
	 \thead{SEI Number} &
	 \thead{SEI Name} &
	 \thead{Table number of encoder controls, if available} \\
	\hline
	\endhead
	 \multicolumn{3}{r}{Continued...}\\
	 \hline
	\endfoot
	 \hline
	\endlastfoot
}{%
	\hline
	\end{longtable}
}

\newenvironment{MacroTable}[1]{%
	\footnotesize
	\def\arraystretch{1.3}
	\clearOptions
	\begin{longtable}{lcp{0.5\textwidth}}
	 \caption{#1} \\
	%\hspace*{12em}&&\hspace*{8em}&\kill
	 \hline
	  \thead{Option} &
	  \thead{Default} &
	  \thead{Description} \\
	 \hline
	\endfirsthead
	 \caption[]{#1 (Continued)} \\
	 \hline
	  \thead{Option} &
	  \thead{Default} &
	  \thead{Description} \\
	 \hline
	\endhead
	 \multicolumn{3}{r}{Continued...}\\
	 \hline
	\endfoot
	 \hline
	\endlastfoot
}{%
	\end{longtable}
}

\title{HM Software Manual}
\author{%
	Frank Bossen
	\email{frank@bossentech.com}
	\and
	David Flynn
	\email{dflynn@blackberry.com}
	\and
  Karl Sharman
	\email{karl.sharman@eu.sony.com}
	\and
	Karsten S\"uhring
	\email{karsten.suehring@hhi.fraunhofer.de}
}

\jctvcmeeting{}
\jctvcdocnum{Software Manual}
\jctvcdocstatus{Software AHG working document}
\jctvcdocpurpose{Information}
\jctvcdocsource{AHG chairs}

\begin{document}
\maketitle
\begin{abstract}
This document is a user manual describing usage of reference software
for the HEVC project. It applies to version 16.5
of the software.
\end{abstract}

\tableofcontents
\listoftables



\section{General Information}
Reference software is being made available to provide a reference
implementation of the HEVC standard being developed by the Joint
Collaborative Team on Video Coding (JCT-VC) regrouping experts from
ITU-T SG 16 and ISO/IEC SC29 WG11. One of the main goals of the
reference software is to provide a basis upon which to conduct
experiments in order to determine which coding tools provide desired
coding performance. It is not meant to be a particularly efficient
implementation of anything, and one may notice its apparent
unsuitability for a particular use. It should not be construed to be a
reflection of how complex a production-quality implementation of a
future HEVC standard would be.

This document aims to provide guidance on the usage of the reference
software. It is widely suspected to be incomplete and suggestions for
improvements are welcome. Such suggestions and general inquiries may be
sent to the general JCT-VC email reflector on
\url{jct-vc@lists.rwth-aachen.de} (registration required).

\subsection*{Bug reporting}
Bugs should be reported on the issue tracker set up at
\url{http://hevc.kw.bbc.co.uk/trac/}

\section{Installation and compilation}
The software may be retrieved from one of the following SVN servers
(mirrored):
\begin{itemize}
\item \url{https://hevc.hhi.fraunhofer.de/svn/svn_HEVCSoftware/}
\item \url{svn://hevc.kw.bbc.co.uk/svn/jctvc-hm/}
\end{itemize}

Table~\ref{tab:project-files} enumerates various project files that are
provided for development environments.

\begin{table}[ht]
\footnotesize
\caption{Available project files}
\label{tab:project-files}
\centering
\begin{tabular}{ll}
\hline
 \thead{Environment} &
 \thead{Location of project file} \\
% Environment          & Location of project file \\
\hline
MS Visual Studio 8   & build/HM_vc8.sln \\
MS Visual Studio 9   & build/HM_vc9.sln \\
MS Visual Studio 10  & build/HM_vc10.sln \\
Xcode                & HM.xcodeproj \\
Linux                & build/linux/makefile \\
\hline
\end{tabular}
\end{table}

For encoding large picture sizes (like UHDTV) it is strongly advised to build 64-bit
binaries and to use a 64-bit OS. This will allow the software to use more than 2GB of RAM.

%%%%
%%%%
%%%%
\section{Using the encoder}
\begin{verbatim}
TAppEncoder 	[--help] [-c config.cfg] [--parameter=value]
\end{verbatim}

\begin{table}[ht]
\footnotesize
\centering
\begin{tabular}{lp{0.5\textwidth}}
\hline
 \thead{Option} &
 \thead{Description} \\
\hline
\texttt{--help} & Prints parameter usage. \\
\texttt{-c} & Defines configuration file to use.  Multiple configuration files
     may be used with repeated --c options. \\
\texttt{--}\emph{parameter}\texttt{=}\emph{value}
    & Assigns value to a given parameter as further described below.
      Some parameters are also supported by shorthand
      "--\em{opt}~\emph{value}". These are shown in brackets after the parameter
      name in the tables of this document\\
\hline
\end{tabular}
\end{table}

Sample configuration files are provided in the cfg/ folder.
Parameters are defined by the last value encountered on the command line.
Therefore if a setting is set via a configuration file, and then a subsequent
command line parameter changes that same setting, the command line parameter
value will be used.

\subsection{GOP structure table}
\label{sec:gop-structure}
Defines the cyclic GOP structure that will be used repeatedly
throughout the sequence. The table should contain GOPSize lines,
named Frame1, Frame2, etc. The frames are listed in decoding
order, so Frame1 is the first frame in decoding order, Frame2 is
the second and so on. Among other things, the table specifies all
reference pictures kept by the decoder for each frame. This
includes pictures that are used for reference for the current
picture as well as pictures that will be used for reference in
the future. The encoder will not automatically calculate which
pictures have to be kept for future references, they must
be specified. Note that some specified reference frames for
pictures encoded in the very first GOP after an IDR frame might
not be available. This is handled automatically by the encoder,
so the reference pictures can be given in the GOP structure table
as if there were infinitely many identical GOPs before the
current one. Each line in the table contains the parameters used
for the corresponding frame, separated by whitespace:

\begin{itemize}
\item[]\textbf{Type}: Slice type, can be either I, P or B.

\item[]\textbf{POC}: Display order of the frame within a GOP, ranging
from 1 to GOPSize.

\item[]\textbf{QPOffset}: QP offset is added to the QP parameter to set
the final QP value to use for this frame.

\item[]\textbf{QPFactor}: Weight used during rate distortion
optimization. Higher values mean lower quality and less bits. Typical
range is between
0.3 and 1.

\item[]\textbf{tcOffsetDiv2}: In-loop deblocking filter parameter tcOffsetDiv2 
is added to the base parameter LoopFilterTcOffset_div2 to set the final tc_offset_div2 
parameter for this picture signalled in the slice segment header. The final 
value of tc_offset_div2 shall be an integer number in the range $-6..6$.

\item[]\textbf{betaOffsetDiv2}: In-loop deblocking filter parameter betaOffsetDiv2 
is added to the base parameter LoopFilterBetaOffset_div2 to set the final beta_offset_div2 
parameter for this picture signalled in the slice segment header. The final 
value of beta_offset_div2 shall be an integer number in the range $-6..6$.

\item[]\textbf{temporal_id}: Temporal layer of the frame. A frame cannot
predict from a frame with a higher temporal id. If a frame with higher
temporal IDs is listed among a frame's reference pictures, it is
not used, but is kept for possible use in future frames.

\item[]\textbf{num_ref_pics_active}: Size of reference picture lists L0
and L1, indicating how many reference pictures in each direction that
are used during coding.

\item[]\textbf{num_ref_pics}: The number of reference pictures kept for
this frame.  This includes pictures that are used for reference for the
current picture as well as pictures that will be used for reference in
the future.

\item[]\textbf{reference_pictures}: A space-separated list of
num_ref_pics integers, specifying the POC of the reference pictures
kept, relative the POC of the current frame. The picture list shall be
ordered, first with negative numbers from largest to smallest, followed
by positive numbers from smallest to largest (e.g. \verb|-1 -3 -5 1 3|).
Note that any pictures not supplied in this list will be discarded and
therefore not available as reference pictures later.

\item[]\textbf{predict}: Defines the value of the syntax element
inter_ref_pic_set_prediction_flag. A value of 0 indicates that the
reference picture set is encoded without inter RPS prediction and the
subsequent parameters deltaRIdx$-1$, deltaRPS, num_ref_idcs and
Reference_idcs are ignored and do not need to be present. A value of 1
indicates that the reference picture set is encoded with inter
prediction RPS using the subsequent parameters deltaRIdx$-1$, deltaRPS,
num_ref_idcs and Reference_idcs in the line. A value of 2 indicates that
the reference picture set is encoded with inter RPS but only the
deltaRIdx$-1$ parameters is needed. The deltaRPS, num_ref_idcs and
Reference_idcs values are automatically derived by the encoder based on
the POC and refPic values of the current line and the RPS pointed to by
the deltaRIdx$-1$ parameters.

\item[]\textbf{deltaRIdx$-1$}: The difference between the index of the
curent RPS and the predictor RPS minus 1.

\item[]\textbf{deltaRPS}: The difference between the POC of the
predictor RPS and POC the current RPS.

\item[]\textbf{num_ref_idcs}: The number of ref_idcs to encode for the
current RPS.  The value is equal to the value of  num_ref_pics of the
predictor RPS plus 1.

\item[]\textbf{reference_idcs}: A space-separated list of num_ref_idcs
integers, specifying the ref idcs of the inter RPS prediction. The value
of ref_idcs may be 0, 1 or 2 indicating that the reference picture is a
reference picture used by the current picture, a reference picture used
for future picture or not a reference picture anymore, respectively. The
first num_ref_pics of ref_idcs correspond to the Reference pictures in
the predictor RPS. The last ref_idcs corresponds to the predictor
picture.
\end{itemize}

For example, consider the coding structure of Figure~\ref{fig:gop-example}.
This coding structure is of size 4. The pictures are listed in decoding
order. Frame1 shall therefore describe picture with $\textrm{POC}=4$. It
references picture 0, and therefore has $-4$ as a reference picture.
Similarly, Frame2 has a POC of 2, and since it references pictures 0 and
4, its reference pictures are listed as \verb|-2 2|. Frame3 is a special
case: even though it only references pictures with POC 0 and 2, it also
needs to include the picture with POC 4, which must be kept in order to
be used as a reference picture in the future. The reference picture list
for Frame3 therefore becomes \verb|-1 1 3|. Frame4 has a POC of 3 and
its list of reference pictures is \verb|-1 1|.

\begin{figure}[h]
\caption{A GOP structure}
\label{fig:gop-example}
\centering
\includegraphics[width=0.7\textwidth]{gop-structure-example}
\end{figure}

Inter RPS prediction may be used for Frame2, Frame3 and Frame4, hence
the predict parameter is set to 1 for these frames. Frame2 uses Frame1
as the predictor hence the deltaRIdx$-1$ is 0.  Similarly for Frame3 and
Frame4 which use Frame2 and Frame3 as predictors, respectively. The
deltaRPS is equal to the POC of the predictor minus the POC of the
current picture, therefore the deltaRPS for Frame2 is $4 -2 = 2$, for
Frame3 is $2 - 1 = 1$ and for Frame4 is $1 - 3 = -2$.

In Frame2, reference pictures with POC 0 and 2 are used, so the
reference idcs for Frame2 are \verb|1 1| indicating that the reference
picture, $-4$, in Frame1 is still a reference picture in Frame2 and
Frame1 is also a reference picture in Frame2. The reference idcs for
Frame3 are \verb|1 1 1|. The first and second “1”s indicating that
the reference pictures "$-2$ $2$" in Frame2 are still reference pictures in
Frame3 and the last “1” indicating that Frame2 is also a reference
picture in Frame3. In Frame 4, the reference idcs are \verb|0 1 1 0|.
The first “0” indicates that the reference pictures “-1” in Frame 3 is
no longer a reference picture in Frame4. The next two “1”s indicate that
the reference pictures “$1$ $3$” are now reference pictures of Frame4.
The final “0” indicates that Frame3 is not a reference picture.

In order to specify this to the encoder, the parameters in
Table~\ref{tab:gop-example} could be used.

\begin{table}[ht]
\footnotesize
\caption{GOP structure example}
\label{tab:gop-example}
\centering
\begin{tabular}{lrrrr}
\hline
 \thead{} &
 \thead{Frame1} &
 \thead{Frame2} &
 \thead{Frame3} &
 \thead{Frame4} \\
\hline
Type                &   P  &    B   &      B   &       B \\
POC                 &   4  &    2   &      1   &       3 \\
QPoffset            &   1  &    2   &      3   &       3 \\
QPfactor            & 0.5  &  0.5   &    0.5   &     0.5 \\
tcOffsetDiv2        &   0  &    1   &      2   &       2 \\  
betaOffsetDiv2      &   0  &    0   &      0   &       0 \\
temporal_id         &   0  &    1   &      2   &       2 \\
num_ref_pics_active &   1  &    1   &      1   &       1 \\
num_ref_pics        &   1  &    2   &      3   &       2 \\
reference_pictures  & $-$4 & $-$2 2 & $-$1 1 3 &  $-$1 1 \\
predict             &   0  &    1   &      1   &       1 \\
deltaRIdx$-$1       &      &    0   &      0   &       0 \\
deltaRPS            &      &    2   &      1   &    $-$2 \\
num_ref_idcs        &      &    2   &      3   &       4 \\
reference_idcs      &      &  1 1   &  1 1 1   & 0 1 1 0 \\
\hline
\end{tabular}
\end{table}

Here, the frames used for prediction have been given higher
quality by assigning a lower QP offset. Also, the non-reference
frames have been marked as belonging to a higher temporal layer,
to make it possible to decode only every other frame. Note: each
line should contain information for one frame, so this
configuration would be specified as:

\begin{verbatim}
Frame1: P 4 1 0.5 0 0 0 1 1 -4 0
Frame2: B 2 2 0.5 1 0 1 1 2 -2 2 1 0 2 2 1 1
Frame3: B 1 3 0.5 2 0 2 1 3 -1 1 3 1 0 1 3 1 1 1
Frame4: B 3 3 0.5 2 0 2 1 2 -1 1 1 0 -2 4 0 1 1 0
\end{verbatim}

The values of deltaRIdx$-1$, deltaRPS, num_ref_idcs and reference
idcs of Frame$K$ can be derived from the POC value of Frame$_K$ and
the POC, num_ref_pics and reference_pictures values of Frame$_M$, where
$K$ is the index of the RPS to be inter coded and the $M$ is the
index of the reference RPS, as follows.

\setlength{\algomargin}{2em}
\begin{algorithm}[h]
\SetKwData{deltaRIdx}{deltaRIdx}
\SetKwData{deltaRPS}{deltaRPS}
\SetKwData{numrefidcs}{num_ref_idcs}
\SetKwData{numrefpics}{num_ref_pics}
\SetKwData{referencepictures}{reference_pictures}
\SetKwData{referenceidcs}{reference_idcs}
\SetKwData{POC}{POC}

$\deltaRIdx_K - 1  \leftarrow  K - M - 1$ \;
$\deltaRPS_K       \leftarrow  \POC_M - \POC_K$ \;
$\numrefidcs_K     \leftarrow  \numrefpics_M + 1$ \;

\For{$j \leftarrow 0$ \KwTo $\numrefpics_M$}{
	\For{$i \leftarrow 0$ \KwTo $\numrefidcs_K$}{
		\eIf{$\referencepictures_{M,j} + \deltaRPS_K == \referencepictures_{K,i}$}{
			\lIf{$\referencepictures_{K,i}$ is used by the current frame}{
				$\referenceidcs_{K,j} = 1$} \;
			\lElse{$\referenceidcs_{K,j} = 2$} \;
		}{
			$\referenceidcs_K[j] = 0$ \;
		}
	}
}

\tcc{$\referencepictures_{M,\numrefpics_M}$ does not exist and is assumed to be 0}
\end{algorithm}

Note: The above (automatic) generation of the inter RPS parameter
values has been integrated into the encoder, and is activated by
the value of predict $= 2$ followed by the value of deltaRIdx$-1$,
only, as described above.



%%%%
%%%%
%%%%
\newgeometry{tmargin=1.6cm,lmargin=1cm,rmargin=1cm,bmargin=1in,nohead}
\subsection{Encoder parameters}

%%
%% File, I/O and source parameters
%%
Shorthand alternatives for the parameter that can be used on the command line are shown in brackets after the parameter name.

\begin{OptionTableNoShorthand}{File, I/O and source parameters.}{tab:fileIO}
\Option{InputFile (-i)} &
%\ShortOption{-i} &
\Default{\NotSet} &
Specifies the input video file.

Video data must be in a raw 4:2:0, or 4:2:2 planar format, 4:4:4 planar format (Y$'$CbCr, RGB or GBR), or in a raw 4:0:0 format. 

Note: When the bit depth of samples is larger than 8, each sample is encoded in
2 bytes (little endian, LSB-justified).
\\

\Option{BitstreamFile (-b)} &
%\ShortOption{-b} &
\Default{\NotSet} &
Specifies the output coded bit stream file.
\\

\Option{ReconFile (-o)} &
%\ShortOption{-o} &
\Default{\NotSet} &
Specifies the output locally reconstructed video file.
\\

\Option{SourceWidth (-wdt)}%
\Option{SourceHeight (-hgt)} &
%\ShortOption{-wdt}%
%\ShortOption{-hgt} &
\Default{0}%
\Default{0} &
Specifies the width and height of the input video in luma samples.
\\

\Option{InputBitDepth}
 &
%\ShortOption{\None} &
\Default{8} &
Specifies the bit depth of the input video.
\\

\Option{MSBExtendedBitDepth} &
%\ShortOption{\None} &
\Default{0} &
Extends the input video by adding MSBs of value 0. When 0, no extension is applied and the InputBitDepth is used.

The MSBExtendedBitDepth becomes the effective file InputBitDepth for subsequent processing.
\\

\Option{InternalBitDepth} &
%\ShortOption{\None} &
\Default{0} &
Specifies the bit depth used for coding. When 0, the setting defaults to the
value of the MSBExtendedBitDepth.

If the input video is a different bit depth to InternalBitDepth, it is
automatically converted by:
\begin{displaymath}
\left\lfloor
 \frac{\mathrm{Pel} * 2^{\mathrm{InternalBitDepth}}}{
     2^{\mathrm{MSBExtendedBitDepth}}}
\right\rfloor
\end{displaymath}

Note: The effect of this option is as if the input video is externally
converted to the MSBExtendedBitDepth and then to the InternalBitDepth
and then coded with this value as InputBitDepth. The codec has no
notion of different bit depths.
\\

\Option{OutputBitDepth} &
%\ShortOption{\None} &
\Default{0} &
Specifies the bit depth of the output locally reconstructed video file.
When 0, the setting defaults to the value of InternalBitDepth.
Note: This option has no effect on the decoding process.
\\

\Option{InputBitDepthC}%
\Option{MSBExtendedBitDepthC}%
\Option{InternalBitDepthC}%
\Option{OutputBitDepthC} &
%\ShortOption{\None} &
\Default{0}%
\Default{0}%
\Default{0}%
\Default{0} &
Specifies the various bit-depths for chroma components.  These only need
to be specified if non-equal luma and chroma bit-depth processing is
required. When 0, the setting defaults to the corresponding non-Chroma value.
\\

\Option{InputColourSpaceConvert} &
%\ShortOption{\None} &
\Default{\NotSet} &
The colour space conversion to apply to input video. Permitted values are:
\par
\begin{tabular}{lp{0.3\textwidth}}
  UNCHANGED & No colour space conversion is applied \\
  YCbCrToYCrCb & Swap the second and third components \\
  YCbCrtoYYY & Set the second and third components to the values in the first \\
  RGBtoGBR & Reorder the three components \\
\end{tabular}
\par
If no value is specified, no colour space conversion is applied. The list may eventually also include RGB to YCbCr or YCgCo conversions.
\\

\Option{SNRInternalColourSpace} &
%\ShortOption{\None} &
\Default{false} &
When this is set true, then no colour space conversion is applied prior to PSNR calculation, otherwise the inverse of InputColourSpaceConvert is applied.
\\

\Option{OutputInternalColourSpace} &
%\ShortOption{\None} &
\Default{false} &
When this is set true, then no colour space conversion is applied to the reconstructed video, otherwise the inverse of InputColourSpaceConvert is applied.
\\

\Option{InputChromaFormat} &
%\ShortOption{\None} &
\Default{420} &
Specifies the chroma format used in the input file. Permitted values (depending on the profile) are 400, 420, 422 or 444.
\\

\Option{ChromaFormatIDC (-cf)} &
%\ShortOption{-cf} &
\Default{0} &
Specifies the chroma format to use for processing. Permitted values (depending on the profile) are 400, 420, 422 or 444; the value of 0 indicates that the value of InputChromaFormat should be used instead.
\\

\Option{MSEBasedSequencePSNR} &
%\ShortOption{\None} &
\Default{false} &
When 0, the PSNR output is a linear average of the frame PSNRs; when 1, additional PSNRs are output which are formed from the average MSE of all the frames. The latter is useful when coding near-losslessly, where occasional frames become lossless.
\\

\Option{PrintFrameMSE} &
%\ShortOption{\None} &
\Default{false} &
When 1, the Mean Square Error (MSE) values of each frame will also be output alongside the default PSNR values.
\\

\Option{PrintSequenceMSE} &
%\ShortOption{\None} &
\Default{false} &
When 1, the Mean Square Error (MSE) values of the entire sequence will also be output alongside the default PSNR values.
\\

\Option{SummaryOutFilename} &
%\ShortOption{\None} &
\Default{false} &
Filename to use for producing summary output file. If empty, do not produce a file.
\\

\Option{SummaryPicFilenameBase} &
%\ShortOption{\None} &
\Default{false} &
Base filename to use for producing summary picture output files. The actual filenames used will have I.txt, P.txt and B.txt appended. If empty, do not produce a file.
\\

\Option{SummaryVerboseness} &
%\ShortOption{\None} &
\Default{false} &
Specifies the level of the verboseness of the text output.
\\

\Option{CabacZeroWordPaddingEnabled} &
%\ShortOption{\None} &
\Default{false} &
When 1, CABAC zero word padding will be enabled. This is currently not the default value for the setting.
\\

\Option{ConformanceWindowMode} &
%\ShortOption{\None} &
\Default{0} &
Specifies how the parameters related to the conformance window are interpreted (cropping/padding).
The following modes are available:
\par
\begin{tabular}{cp{0.43\textwidth}}
0 & No cropping / padding \\
1 & Automatic padding to the next minimum CU size \\
2 & Padding according to parameters HorizontalPadding and VerticalPadding \\
3 & Cropping according to parameters ConfWinLeft, ConfWinRight, ConfWinTop and ConfWinBottom \\
\end{tabular}
\\

\Option{HorizontalPadding (-pdx)}%
\Option{VerticalPadding (-pdy)} &
%\ShortOption{-pdx}%
%\ShortOption{-pdy} &
\Default{0} &
Specifies the horizontal and vertical padding to be applied to the input
video in luma samples when ConformanceWindowMode is 2.  Must be a multiple of
the chroma resolution (e.g. a multiple of two for 4:2:0).
\\

\Option{ConfWinLeft}%
\Option{ConfWinRight}%
\Option{ConfWinTop}%
\Option{ConfWinBottom} &
%\ShortOption{\None} &
\Default{0} &
Specifies the horizontal and vertical cropping to be applied to the
input video in luma samples when ConformanceWindowMode is 3. 
Must be a multiple of the chroma resolution (e.g. a multiple of
two for 4:2:0).
\\

\Option{FrameRate (-fr)} &
%\ShortOption{-fr} &
\Default{0} &
Specifies the frame rate of the input video.

Note: This option only affects the reported bit rates.
\\

\Option{FrameSkip (-fs)} &
%\ShortOption{-fs} &
\Default{0} &
Specifies a number of frames to skip at beginning of input video file.
\\

\Option{FramesToBeEncoded (-f)} &
%\ShortOption{-f} &
\Default{0} &
Specifies the number of frames to be encoded. When 0, all frames are coded.
\\

\Option{FieldCoding} &
%\ShortOption{\None} &
\Default{false} &
When 1, indicates that field-based coding is to be applied.
\\

\Option{TopFieldFirst (-Tff)} &
%\ShortOption{\None} &
\Default{0} &
Indicates the order of the fields packed into the input frame. When 1, the top field is temporally first.
\\

\Option{ClipInputVideoToRec709Range} &
%\ShortOption{\None} &
\Default{0} &
If 1 then clip input video to the Rec. 709 Range on loading when InternalBitDepth is less than MSBExtendedBitDepth.
\\

\Option{ClipOutputVideoToRec709Range} &
%\ShortOption{\None} &
\Default{0} &
If 1 then clip output video to the Rec. 709 Range on saving when OutputBitDepth is less than InternalBitDepth.
\\

\Option{EfficientFieldIRAPEnabled} &
%\ShortOption{\None} &
\Default{1} &
Enable to code fields in a specific, potentially more efficient, order.
\\

\Option{HarmonizeGopFirstFieldCoupleEnabled} &
%\ShortOption{\None} &
\Default{1} &
Enables harmonization of Gop first field couple.
\\

\end{OptionTableNoShorthand}

%%
%% profile, level and conformance options
%%
\begin{OptionTableNoShorthand}{Profile and level parameters}{tab:profile}
\Option{Profile} &
%\ShortOption{\None} &
\Default{none} &
Specifies the profile to which the encoded bitstream complies.

Valid HEVC Ver. 1 values are: none, main, main10, main-still-picture

Valid HEVC Ver. 2 (RExt) values are: main-RExt, high-throughput-RExt,
monochrome, monochrome12, monochrome16, main12, main_422_10,
main_422_12, main_444, main_444_10, main_444_12, main_444_16,
main_intra, main_10_intra, main_12_intra, main_422_10_intra, main_422_12_intra,
main_444_intra, main_444_10_intra, main_444_12_intra, main_444_16_intra.

When main-RExt is specified, the constraint flags are either manually specified, or calculated via the other supplied settings.

Compatibility flags are automatically determined according to the profile.
NB: There is currently only limited validation that the encoder configuration complies with the profile, level and tier constraints.
\\

\Option{Level} &
%\ShortOption{\None} &
\Default{none} &
Specifies the level to which the encoded bitstream complies.
Valid values are: none, 1, 2, 2.1, 3, 3.1, 4, 4.1, 5, 5.1, 5.2, 6, 6.1, 6.2, 8.5

NB: There is currently only limited validation that the encoder configuration complies with the profile, level and tier constraints.
\\

\Option{Tier} &
%\ShortOption{\None} &
\Default{main} &
Specifies the level tier to which the encoded bitsream complies.
Valid values are: main, high.

NB: There is currently only limited validation that the encoder configuration complies with the profile, level and tier constraints.
\\

\Option{MaxBitDepthConstraint} &
%\ShortOption{\None} &
\Default{0} &
For --profile=main-RExt, specifies the value to use to derive the general_max_bit_depth constraint flags for RExt profiles; when 0, use $\max(InternalBitDepth, InternalBitDepthC)$
\\

\Option{MaxChromaFormatConstraint} &
%\ShortOption{\None} &
\Default{0} &
For --profile=main-RExt, specifies the chroma-format to use for the general profile constraints for RExt profiles; when 0, use the value of ChromaFormatIDC.
\\

\Option{IntraConstraintFlag} &
%\ShortOption{\None} &
\Default{false} &
For --profile=main-RExt, specifies the value of general_intra_constraint_flag to use for RExt profiles.
\\

\Option{OnePictureOnlyConstraintFlag} &
%\ShortOption{\None} &
\Default{false} &
For --profile=main-RExt, specifies the value of general_one_picture_only_constraint_flag to use for RExt profiles.
\\

\Option{LowerBitRateConstraintFlag} &
%\ShortOption{\None} &
\Default{true} &
Specifies the value of general_lower_bit_constraint_flag to use for RExt profiles.
\\

\Option{ProgressiveSource} &
%\ShortOption{\None} &
\Default{false} &
Specifies the value of general_progressive_source_flag
\\

\Option{InterlacedSource} &
%\ShortOption{\None} &
\Default{false} &
Specifies the value of general_interlaced_source_flag 
\\

\Option{NonPackedSource} &
%\ShortOption{\None} &
\Default{false} &
Specifies the value of general_non_packed_constraint_flag
\\

\Option{FrameOnly} &
%\ShortOption{\None} &
\Default{false} &
Specifies the value of general_frame_only_constraint_flag
\\

\end{OptionTableNoShorthand}


%%
%% Unit definition parameters
%%

\begin{OptionTableNoShorthand}{Unit definition parameters}{tab:unit}
\Option{MaxCUWidth} &
%\ShortOption{\None} &
\Default{64} &
Defines the maximum CU width.
\\

\Option{MaxCUHeight} &
%\ShortOption{\None} &
\Default{64} &
Defines the maximum CU height.
\\

\Option{MaxCUSize (-s)} &
%\ShortOption{\None} &
\Default{64} &
Defines the maximum CU size.
\\

\Option{MaxPartitionDepth (-h)} &
%\ShortOption{-h} &
\Default{4} &
Defines the depth of the CU tree.
\\

\Option{QuadtreeTULog2MaxSize} &
%\ShortOption{\None} &
\Default{6 \\ ($= \mathrm{log}_2(64)$)} &
Defines the Maximum TU size in logarithm base 2.
\\

\Option{QuadtreeTULog2MinSize} &
%\ShortOption{\None} &
\Default{2 \\ ($= \mathrm{log}_2(4)$)} &
Defines the Minimum TU size in logarithm base 2.
\\

\Option{QuadtreeTUMaxDepthIntra} &
%\ShortOption{\None} &
\Default{1} &
Defines the depth of the TU tree for intra CUs.
\\

\Option{QuadtreeTUMaxDepthInter} &
%\ShortOption{\None} &
\Default{2} &
Defines the depth of the TU tree for inter CUs.
\\

\end{OptionTableNoShorthand}


%%
%% Coding structure parameters
%%

\begin{OptionTableNoShorthand}{Coding structure parameters}{tab:coding-structure}
\Option{IntraPeriod (-ip)} &
%\ShortOption{-ip} &
\Default{$-1$} &
Specifies the intra frame period.
A value of $-1$ implies an infinite period.
\\

\Option{DecodingRefreshType (-dr)} &
%\ShortOption{-dr} &
\Default{0} &
Specifies the type of decoding refresh to apply at the intra frame period
picture.
\par
\begin{tabular}{cp{0.45\textwidth}}
0 & Applies an I picture (not a intra random access point). \\
1 & Applies a CRA intra random access point (open GOP). \\
2 & Applies an IDR intra random access point (closed GOP). \\
3 & Use recovery point SEI messages to indicate random access. \\
\end{tabular}
\\

\Option{GOPSize (-g)} &
%\ShortOption{-g} &
\Default{1} &
Specifies the size of the cyclic GOP structure.
\\

\Option{Frame\emph{N}} &
%\ShortOption{\None} &
\Default{\NotSet} &
Multiple options that define the cyclic GOP structure that will be used
repeatedly throughout the sequence.  The table should contain GOPSize
elements.
\par
See section~\ref{sec:gop-structure} for further details.
\\
\end{OptionTableNoShorthand}

%%
%% Motion estimation parameters
%%

\begin{OptionTableNoShorthand}{Motion estimation parameters}{tab:motion-estimation}
\Option{FastSearch} &
%\ShortOption{\None} &
\Default{1} &
Enables or disables the use of a fast motion search.
\par
\begin{tabular}{cp{0.45\textwidth}}
 0 & Full search method \\
 1 & Fast search method \\
 2 & Previous motion vector fast search method \\
\end{tabular}
\\

\Option{SearchRange (-sr)} &
%\ShortOption{-sr} &
\Default{96} &
Specifies the search range used for motion estimation.

Note: the search range is defined around a predictor. Motion vectors
derived by the motion estimation may thus have values larger than the
search range.
\\

\Option{BipredSearchRange} &
%\ShortOption{\None} &
\Default{4} &
Specifies the search range used for bi-prediction refinement in motion
estimation.
\\

\Option{ClipForBiPredMEEnabled} &
%\ShortOption{\None} &
\Default{0} &
Enables clipping in the Bi-Pred ME, which prevents values over- or under-flowing. It is usually disabled to reduce encoder run-time.
\\

\Option{FastMEAssumingSmootherMVEnabled} &
%\ShortOption{\None} &
\Default{0} &
Enables fast ME assuming a smoother MV.
\\

\Option{HadamardME} &
%\ShortOption{\None} &
\Default{true} &
Enables or disables the use of the Hadamard transform in fractional-pel motion
estimation.
\par
\begin{tabular}{cp{0.45\textwidth}}
 0 & SAD for cost estimation \\
 1 & Hadamard for cost estimation \\
\end{tabular}
\\

\Option{ASR} &
%\ShortOption{\None} &
\Default{false} &
Enables or disables the use of adaptive search ranges, where the motion
search range is dynamically adjusted according to the POC difference
between the current and the reference pictures.
\begin{displaymath}
\resizebox{\hsize}{!}{$
\mathrm{SearchRange}’ = \mathrm{Round}\left(
                          \mathrm{SearchRange}
                          * \mathrm{ADAPT\_SR\_SCALE}
                          * \frac{\mathrm{abs}(
                                 \mathrm{POCcur} - \mathrm{POCref} )}{
                                 \mathrm{RateGOPSize}}\right)
$}
\end{displaymath}
\\

\Option{MaxNumMergeCand} &
%\ShortOption{\None} &
\Default{5} &
Specifies the maximum number of merge candidates to use.
\\

\Option{DisableIntraInInter} &
%\ShortOption{\None} &
\Default{0} &
Flag to disable intra PUs in inter slices.
\\

\end{OptionTableNoShorthand}


%%
%% Mode decision parameters
%%

\begin{OptionTableNoShorthand}{Mode decision parameters}{tab:mode-decision}
\Option{LambdaModifier$N$ (-LM$N$)} &
%\ShortOption{-LM$N$} &
\Default{1.0} &
Specifies a value that is multiplied with the Lagrange multiplier
$\lambda$, for use in the rate-distortion optimised cost calculation
when encoding temporal layer~$N$.
\par
$N$ may be in the range 0 (inclusive) to 7 (exclusive).
\\

\Option{ECU} &
%\ShortOption{\None} &
\Default{false} &
Enables or disables the use of early CU determination.  When enabled, skipped CUs will not be split further.
\\

\Option{CFM} &
%\ShortOption{\None} &
\Default{false} &
Enables or disables the use of Cbf-based fast encoder mode.  When enabled, once a 2Nx2N CU has been evaluated, if the RootCbf is 0, further PU splits will not be evaluated.
\\

\Option{ESD} &
%\ShortOption{\None} &
\Default{false} &
Enables or disables the use of early skip detection.  When enabled, the skip mode will be tested before any other.
\\

\Option{FEN} &
%\ShortOption{\None} &
\Default{false} &
Enables or disables the use of fast encoder mode.  When enabled,
the following occurs:
\begin{itemize}
\item In the SAD computation for blocks having size larger than 8, only
      the lines of even rows in the block are considered.
\item The number of iterations used in the bi-directional motion vector
      refinement in the motion estimation process is reduced from 4 to 1.
\end{itemize}
\\

\Option{FDM} &
%\ShortOption{\None} &
\Default{true} &
Enables or disables the use of fast encoder decisions for 2Nx2N merge
mode.  When enabled, the RD cost for the merge mode of the current
candidate is not evaluated if the merge skip mode was the best merge
mode for one of the previous candidates.
\\

\Option{RDpenalty} &
%\ShortOption{\None} &
\Default{0} &
RD-penalty for 32x32 TU for intra in non-intra slices.
Enabling this parameter can reduce the visibility of CU boundaries in the coded picture.
\par
\begin{tabular}{cp{0.45\textwidth}}
 0 & No RD-penalty \\
 1 & RD-penalty \\
 2 & Maximum RD-penalty (no 32x32 TU)\\
\end{tabular}
\\

\end{OptionTableNoShorthand}

%%
%% Quantization parameters
%%
\begin{OptionTableNoShorthand}{Quantization parameters}{tab:quantization}
\Option{QP (-q)} &
%\ShortOption{-q} &
\Default{30.0} &
Specifies the base value of the quantization parameter. If it is non-integer, the QP is switched once during encoding.
\\

\Option{CbQpOffset (-cbqpofs)}%
\Option{CrQpOffset (-crqpofs)} &
%\ShortOption{-cbqpofs}%
%\ShortOption{-crqpofs} &
\Default{0}%
\Default{0} &
Global offset to apply to the luma QP to derive the QP of Cb and Cr
respectively.  These options correspond to the values of cb_qp_offset
and cr_qp_offset, that are transmitted in the PPS.  Valid values are in
the range $[-12, 12]$.
\\

\Option{MaxCuDQPDepth (-dqd)} &
%\ShortOption{\None} &
\Default{0} &
Defines maximum depth of a minimum CuDQP for sub-LCU-level delta QP.
MaxCuDQPDepth shall be greater than or equal to SliceGranularity.
\\

\Option{RDOQ} &
%\ShortOption{\None} &
\Default{true} &
Enables or disables rate-distortion-optimized quantization for transformed TUs.
\\

\Option{RDOQTS} &
%\ShortOption{\None} &
\Default{true} &
Enables or disables rate-distortion-optimized quantization for transform-skipped TUs.
\\

\Option{SelectiveRDOQ} &
%\ShortOption{\None} &
\Default{false} &
Enables or disables selective rate-distortion-optimized quantization.
A simple quantization is use to pre-analyze, whether to bypass the RDOQ process or not. 
If all the coefficients are quantized to 0, the RDOQ process is bypassed. 
Otherwise, the RDOQ process is performed as usual.
\\

\Option{DeltaQpRD (-dqr)} &
%\ShortOption{-dqr} &
\Default{0} &
Specifies the maximum QP offset at slice level for multi-pass slice
encoding.  When encoding, each slice is tested multiple times by using
slice QP values in the range $[-\mathrm{DeltaQpRD}, \mathrm{DeptaQpRD}]$,
and the best QP value is chosen as the slice QP.
\\

\Option{MaxDeltaQP (-d)} &
%\ShortOption{-d} &
\Default{0} &
Specifies the maximum QP offset at the largest coding unit level for
the block-level adaptive QP assignment scheme. In the encoder, each
largest coding unit is tested multiple times by using the QP values in
the range $[-\mathrm{MaxDeltaQP}, \mathrm{MaxDeltaQP}]$, and the best QP
value is chosen as the QP value of the largest coding unit.
\\

\Option{dQPFile (-m)} &
%\ShortOption{-m} &
\Default{\NotSet} &
Specifies a file containing a list of QP deltas. The $n$-th line
(where $n$ is 0 for the first line) of this file corresponds to the QP
value delta for the picture with POC value $n$.
\\

\Option{AdaptiveQp (-aq)} &
%\ShortOption{-aq} &
\Default{false} &
Enable or disable QP adaptation based upon a psycho-visual model.
\\

\Option{MaxQPAdaptationRange (-aqr)} &
%\ShortOption{-aqps} &
\Default{6} &
Specifies the maximum QP adaptation range.
\\

\Option{AdaptiveQpSelection (-aqps)} &
%\ShortOption{-aqps} &
\Default{false} &
Specifies whether QP values for non-I frames will be calculated on the
fly based on statistics of previously coded frames.
\\

\Option{RecalculateQP...} \Option{AccordingToLambda} &
%\ShortOption{\None} &
\Default{false} &
Recalculate QP values according to lambda values. Do not suggest to be enabled in all intra case.
\\

\Option{ScalingList} &
%\ShortOption{\None} &
\Default{0} &
Controls the specification of scaling lists:
\par
\begin{tabular}{cp{0.45\textwidth}}
 0 & Scaling lists are disabled \\
 1 & Use default scaling lists \\
 2 & Scaling lists are specified in the file indicated by ScalingListFile \\
\end{tabular}
\\

\Option{ScalingListFile} &
%\ShortOption{\None} &
\Default{\NotSet} &
When ScalingList is set to 2, this parameter indicates the name of the file, which contains the defined scaling lists.
If ScalingList is set to 2 and this parameter is an empty string, information on the format of the scaling list file
is output and the encoder stops.
\\

\Option{MaxCUChromaQpAdjustmentDepth} &
%\ShortOption{\None} &
\Default{-1} &
Specifies the maximum depth for CU chroma QP adjustment; if negative, CU chroma QP adjustment is disabled.
\\

\end{OptionTableNoShorthand}


%%
%% Slice coding parameters
%%
\begin{OptionTableNoShorthand}{Slice coding parameters}{tab:slice-coding}
%\Option{SliceGranularity} &
%\ShortOption{\None} &
%\Default{0} &
%Determines the depth in an LCU at which slices may begin and end.
%\par
%\begin{tabular}{cp{0.45\textwidth}}
% 0   & Slice addresses are LCU aligned \\
% $1 \leq n \leq 3$
%     & Slice start addresses are aligned to CUs at depth $n$ \\
%\end{tabular}
%
%Note: The smallest permissible alignment is 16x16 CUs.
%Values of $n$ must satisfy this constraint, for example, with a 64x64
%LCU, $n$ must be less than or equal to 2.
%\\

\Option{SliceMode} &
%\ShortOption{\None} &
\Default{0} &
Controls the slice partitioning method in conjunction with
SliceArgument.
\par
\begin{tabular}{cp{0.45\textwidth}}
 0 & Single slice \\
 1 & Maximum number of CTUs per slice \\
 2 & Maximum number of bytes per slice \\
 3 & Maximum number of tiles per slice \\
\end{tabular}
\\

\Option{SliceArgument} &
%\ShortOption{\None} &
\Default{\NotSet} &
Specifies the maximum number of CTUs, bytes or tiles in a slice depending on the
SliceMode setting.
\\

\Option{SliceSegmentMode} &
%\ShortOption{\None} &
\Default{0} &
Enables (dependent) slice segment coding in conjunction with 
SliceSegmentArgument.
\par
\begin{tabular}{cp{0.45\textwidth}}
 0 & Single slice \\
 1 & Maximum number of CTUs per slice segment\\
 2 & Maximum number of bytes per slice segment\\
 3 & Maximum number of tiles per slice segment\\
\end{tabular}
\\

\Option{SliceSegmentArgument} &
%\ShortOption{\None} &
\Default{\NotSet} &
Defines the maximum number of CTUs, bytes or tiles a slice segment
depending on the SliceSegmentMode setting.
\\

\Option{WaveFrontSynchro} &
%\ShortOption{\None} &
\Default{false} &
Enables the use of specific CABAC probabilities synchronization at the
beginning of each line of CTBs in order to produce a bitstream that can
be encoded or decoded using one or more cores.
\\

\Option{TileUniformSpacing} &
%\ShortOption{\None} &
\Default{false} &
Controls the mode used to determine per row and column tile sizes.
\par
\begin{tabular}{cp{0.45\textwidth}}
 0 & Each tile column width and tile row height is explicitly set
     by TileColumnWidthArray and TileRowHeightArray respectively \\
 1 & Tile columns and tile rows are uniformly spaced. \\
\end{tabular}
\\

\Option{NumTileColumnsMinus1}%
\Option{NumTileRowsMinus1} &
%\ShortOption{\None} &
\Default{0} &
Specifies the tile based picture partitioning geometry as
$\mathrm{NumTileColumnsMinus1} + 1 \times \mathrm{NumTileRowsMinus1} + 1$
columns and rows.
\\

\Option{TileColumnWidthArray}%
\Option{TileRowHeightArray} &
%\ShortOption{\None} &
\Default{\NotSet} &
Specifies a space or comma separated list of widths and heights,
respectively, of each tile column or tile row.  The first value in the
list corresponds to the leftmost tile column or topmost tile row.
\\
\end{OptionTableNoShorthand}



%%
%% Deblocking filter parameters
%%
\begin{OptionTableNoShorthand}{Deblocking filter parameters}{tab:deblocking-filter}
\Option{LoopFilterDisable} &
%\ShortOption{\None} &
\Default{false} &
Enables or disables the in-loop deblocking filter.
\\

\Option{LFCrossSliceBoundaryFlag} &
%\ShortOption{\None} &
\Default{true} &
Enables or disables the use of in-loop filtering across slice
boundaries.
\\

\Option{LoopFilterOffsetInPPS}&
%\ShortOption{\None}&
\Default{false}&
If enabled, the in-loop deblocking filter control parameters are sent in PPS. 
Otherwise, the in-loop deblocking filter control parameters are sent in the slice segment header.
If deblocking filter parameters are sent in PPS, the same values of deblocking filter parameters 
are used for all pictures in the sequence (i.e. deblocking parameter = base parameter value).  
If deblocking filter parameters are sent in the slice segment header, varying deblocking filter 
parameters can be specified by setting parameters tcOffsetDiv2 and betaOffsetDiv2 in the GOP structure table. 
In this case, the final value of the deblocking filter parameter sent for a certain GOP picture is equal to 
(base parameter + GOP parameter for this picture). Intra-pictures use the base parameters values.
\\

\Option{LoopFilterTcOffset_div2}&
%\ShortOption{\None}&
\Default{0}&
Specifies the base value for the in-loop deblocking filter parameter tc_offset_div2. The final value of tc_offset_div2 
shall be an integer number in the range $-6..6$.
\\

\Option{LoopFilterBetaOffset_div2}&
%\ShortOption{\None}&
\Default{0}&
Specifies the base value for the in-loop deblocking filter parameter beta_offset_div2. The final value of beta_offset_div2 
shall be an integer number in the range $-6..6$.
\\

\Option{DeblockingFilterMetric}&
%\ShortOption{\None}&
\Default{false}&
Enables or disables the use of a deblocking filter metric to evaluate the suitability of deblocking. If enabled then
LoopFilterOffsetInPPS and LoopFilterDisable must be 0.
\\

\Option{LFCrossSliceBoundaryFlag}&
%\ShortOption{\None}&
\Default{true}&
Enables or disables the use of a deblocking across tile boundaries.
\\

\end{OptionTableNoShorthand}



%%
%% Coding tools parameters
%%

\begin{OptionTableNoShorthand}{Coding tools parameters}{tab:coding-tools}

\Option{AMP} &
%\ShortOption{\None} &
\Default{true} &
Enables or disables the use of asymmetric motion partitions.
\\

\Option{SAO} &
%\ShortOption{\None} &
\Default{true} &
Enables or disables the sample adaptive offset (SAO) filter.
\\

\Option{TestSAODisableAtPictureLevel} &
%\ShortOption{\None} &
\Default{false} &
Enables the testing of disabling SAO at the picture level after having analysed all blocks.
\\

\Option{SaoEncodingRate} &
%\ShortOption{\None} &
\Default{0.75} &
When >0 SAO early picture termination is enabled for luma and chroma.
\\

\Option{SaoEncodingRateChroma} &
%\ShortOption{\None} &
\Default{0.5} &
The SAO early picture termination rate to use for chroma (when m_SaoEncodingRate is >0). If <=0, use results for luma.
\\

\Option{SAOLcuBoundary} &
%\ShortOption{\None} &
\Default{false} &
Enables or disables SAO parameter estimation using non-deblocked pixels
for LCU bottom and right boundary areas.
\\

\Option{ConstrainedIntraPred} &
%\ShortOption{\None} &
\Default{false} &
Enables or disables constrained intra prediction.  Constrained intra
prediction only permits samples from intra blocks in the same slice as the
current block to be used for intra prediction.
\\

\Option{FastUDIUseMPMEnabled} &
%\ShortOption{\None} &
\Default{true} &
If enabled, adapt intra direction search, accounting for MPM
\\

\Option{FastMEForGenBLowDelayEnabled} &
%\ShortOption{\None} &
\Default{true} &
If enabled use a fast ME for generalised B Low Delay slices
\\

\Option{UseBLambdaForNonKeyLowDelayPictures} &
%\ShortOption{\None} &
\Default{true} &
Enables use of B-Lambda for non-key low-delay pictures
\\

\Option{TransquantBypassEnableFlag} &
%\ShortOption{\None} &
\Default{false} &
Enables or disables the ability to bypass the transform,
quantization and filtering stages at CU level.
This option corresponds to the value of
transquant_bypass_enable_flag that is transmitted in the PPS.

See CUTransquantBypassFlagForce for further details.
\\

\Option{CUTransquantBypassFlagForce} &
%\ShortOption{\None} &
\Default{0} &
Controls the per CU transformation, quantization and filtering
mode decision.
This option controls the value of the per CU cu_transquant_bypass_flag.
\par
\begin{tabular}{cp{0.45\textwidth}}
 0 & Bypass is searched on a CU-by-CU basis and will be used if the cost is lower than not bypassing. \\
 1 & Bypass is forced for all CUs. \\
\end{tabular}

This option has no effect if TransquantBypassEnableFlag is disabled.
\\

\Option{PCMEnabledFlag} &
%\ShortOption{\None} &
\Default{false} &
Enables or disables the use of PCM. The encoder will use cost measures on a CU-by-CU basis to determine if PCM mode is to be applied.
\\

\Option{PCMLog2MaxSize} &
%\ShortOption{\None} &
\Default{5 \\ ($= \mathrm{log}_2(32)$)} &
Specifies log2 of the maximum PCM block size. When PCM is enabled, the
PCM mode is available for 2Nx2N intra PUs smaller than or equal to the
specified maximum PCM block size
\\

\Option{PCMLog2MinSize} &
%\ShortOption{\None} &
\Default{3} &
Specifies log2 of the minimum PCM block size. When PCM is enabled, the
PCM mode is available for 2Nx2N intra PUs larger than or equal to the
specified minimum PCM block size.
\par
When larger than PCMLog2MaxSize, PCM mode is not used.
\\

\Option{PCMInputBitDepthFlag} &
%\ShortOption{\None} &
\Default{true} &
If enabled specifies that PCM sample bit-depth is set equal to
InputBitDepth. Otherwise, it specifies that PCM sample bit-depth is set
equal to InternalBitDepth.
\\

\Option{PCMFilterDisableFlag} &
%\ShortOption{\None} &
\Default{false} &
If enabled specifies that loop-filtering on reconstructed samples of PCM
blocks is skipped. Otherwise, it specifies that loop-filtering on
reconstructed samples of PCM blocks is not skipped.
% 0 = (loop-filtering is not skipped for PCM samples).
\\

\Option{WeightedPredP (-wpP)} &
%\ShortOption{-wpP} &
\Default{false} &
Enables the use of weighted prediction in P slices.
\\

\Option{WeightedPredB (-wpB)} &
%\ShortOption{-wpB} &
\Default{false} &
Enables the use of weighted prediction in B slices.
\\

\Option{Log2ParallelMergeLevel} &
%\ShortOption{\None} &
\Default{2} &
Defines the PPS-derived Log2ParMrgLevel variable.
\\

\Option{SignHideFlag (-SBH)} &
%\ShortOption{-SBH} &
\Default{true} &
If enabled specifies that for each 4x4 coefficient group for which the
number of coefficients between the first nonzero coefficient and the
last nonzero coefficient along the scanning line exceeds 4, the sign bit
of the first nonzero coefficient will not be directly transmitted in the
bitstream, but may be inferred from the parity of the sum of all nonzero
coefficients in the current coefficient group.
\\

\Option{StrongIntraSmoothing (-sis)} &
%\ShortOption{-sis} &
\Default{true} &
If enabled specifies that for 32x32 intra prediction block, the intra smoothing
when applied is either the 1:2:1 smoothing filter or a stronger bi-linear 
interpolation filter.  Key reference sample values are tested and if the criteria 
is satisfied, the stronger intra smoothing filter is applied.
If disabled, the intra smoothing filter when applied is the 1:2:1 smoothing filter.
\\

\Option{TMVPMode} &
%\ShortOption{\None} &
\Default{1} &
Controls the temporal motion vector prediction mode.
\par
\begin{tabular}{cp{0.45\textwidth}}
  0 & Disabled for all slices. \\
  1 & Enabled for all slices. \\
  2 & Disabled only for the first picture of each GOPSize. \\
\end{tabular}
\\

\Option{TransformSkip} &
%\ShortOption{\None} &
\Default{false} &
Enables or disables transform-skipping mode decision.
\\

\Option{TransformSkipFast} &
%\ShortOption{\None} &
\Default{false} &
Enables or disables reduced testing of the transform-skipping mode
decision for chroma TUs.  When enabled, no RDO search is performed for
chroma TUs, instead they are transform-skipped if the four corresponding
luma TUs are also skipped.
\par
This option has no effect if TransformSkip is disabled.
\\
\end{OptionTableNoShorthand}

%%
%% Rate control parameters
%%

\begin{OptionTableNoShorthand}{Rate control parameters}{tab:rate-control}

\Option{RateControl} &
%\ShortOption{\None} &
\Default{false} &
Rate control: enables rate control or not.
\\

\Option{TargetBitrate} &
%\ShortOption{\None} &
\Default{0} &
Rate control: target bitrate, in bps.
\\

\Option{KeepHierarchicalBit} &
%\ShortOption{\None} &
\Default{0} &
Rate control: 0: equal bit allocation among pictures;
1: fix ratio hierarchical bit allocation; 2: adaptive hierarchical ratio bit allocation.
It is suggested to enable hierarchical bit allocation for hierarchical-B coding structure.
\\

\Option{LCULevelRateControl} &
%\ShortOption{\None} &
\Default{true} &
Rate control: true: LCU level RC; false: picture level RC.
\\

\Option{RCLCUSeparateModel} &
%\ShortOption{\None} &
\Default{true} &
Rate control: use LCU level separate R-lambda model or not.
When LCULevelRateControl is equal to false, this parameter is meaningless.
\\

\Option{InitialQP} &
%\ShortOption{\None} &
\Default{0} &
Rate control: initial QP value for the first picture.
0 to auto determine the initial QP value.
\\

\Option{RCForceIntraQP} &
%\ShortOption{\None} &
\Default{false} &
Rate control: force intra QP to be equal to initial QP or not.
\\
\end{OptionTableNoShorthand}

%%
%% VUI parameters
%%
\begin{OptionTableNoShorthand}{VUI parameters}{tab:VUI}
\Option{VuiParametersPresent (-vui)} &
\Default{false} &
Enable generation of vui_parameters().
\\
\Option{AspectRatioInfoPresent} &
\Default{false} &
Signals whether aspect_ratio_idc is present.
\\
\Option{AspectRatioIdc} &
\Default{0} &
aspect_ratio_idc
\\
\Option{SarWidth} &
\Default{0} &
Specifies the horizontal size of the sample aspect ratio.
\\
\Option{SarHeight} &
\Default{0} &
Specifies the vertical size of the sample aspect ratio.
\\
\Option{OverscanInfoPresent} &
\Default{false} &
Signals whether overscan_info_present_flag is present.
\\
\Option{OverscanAppropriate} &
\Default{false} &
Indicates whether cropped decoded pictures are suitable for display using overscan.
\par
\begin{tabular}{cp{0.45\textwidth}}
  0 & Indicates that the decoded pictures should not be displayed using overscan. \\
  1 & Indicates that the decoded pictures may be displayed using overscan. \\
\end{tabular}
\\
\Option{VideoSignalTypePresent} &
\Default{false} &
Signals whether video_format, video_full_range_flag, and colour_description_present_flag are present.
\\
\Option{VideoFormat} &
\Default{5} &
Indicates representation of pictures.
\\
\Option{VideoFullRange} &
\Default{false} &
Indicates the black level and range of luma and chroma signals.
\par
\begin{tabular}{cp{0.45\textwidth}}
  0 & Indicates that the luma and chroma signals are to be scaled prior to display. \\
  1 & Indicates that the luma and chroma signals are not to be scaled prior to display. \\
\end{tabular}
\\
\Option{ColourDescriptionPresent} &
\Default{false} &
Signals whether colour_primaries, transfer_characteristics and matrix_coefficients are present.
\\
\Option{ColourPrimaries} &
\Default{2} &
Indicates chromaticity coordinates of the source primaries.
\\
\Option{TransferCharateristics} &
\Default{2} &
Indicates the opto-electronic transfer characteristics of the source.
\\
\Option{MatrixCoefficients} &
\Default{2} &
Describes the matrix coefficients used in deriving luma and chroma from RGB primaries.
\\
\Option{ChromaLocInfoPresent} &
\Default{false} &
Signals whether chroma_sample_loc_type_top_field and chroma_sample_loc_type_bottom_field are present.
\\
\Option{ChromaSampleLocTypeTopField} &
\Default{0} &
Specifies the location of chroma samples for top field.
\\
\Option{ChromaSampleLocTypeBottomField} &
\Default{0} &
Specifies the location of chroma samples for bottom field.
\\
\Option{NeutralChromaIndication} &
\Default{false} &
Indicates that the value of all decoded chroma samples is equal to 1<<(BitDepthCr-1).
\\

\Option{DefaultDisplayWindowFlag} &
\Default{flag} &
Indicates the presence of the Default Window parameters.
\par
\begin{tabular}{cp{0.45\textwidth}}
false & Disabled \\
true & Enabled \\
\end{tabular}
\\

\Option{DefDispWinLeftOffset}%
\Option{DefDispWinRightOffset}%
\Option{DefDispWinTopOffset}%
\Option{DefDispWinBottomOffset} &
\Default{0} &
Specifies the horizontal and vertical offset to be applied to the
input video from the conformance window in luma samples.
Must be a multiple of the chroma resolution (e.g. a multiple of two for 4:2:0).
\\

\Option{FrameFieldInfoPresentFlag} &
\Default{false} &
Specificies the value of the VUI syntax element `frame_field_info_present_flag', which indicates that pic_struct and field coding related values are present in picture timing SEI messages.
\\

\Option{PocProportionalToTimingFlag} &
\Default{false} &
Specificies the value of the VUI syntax element `vui_poc_proportional_to_timing_flag', which indicates that the POC value is proportional to the output time with respect to the first picture in the CVS.
\\

\Option{NumTicksPocDiffOneMinus} &
\Default{0} &
Specificies the value of the VUI syntax element `vui_num_ticks_poc_diff_one_minus1', which specifies the number of clock ticks corresponding to a difference of picture order count values equal to 1, and is used only when PocProportionalToTimingFlag is true.
\\

\Option{BitstreamRestriction} &
\Default{false} &
Signals whether bitstream restriction parameters are present.
\\
\Option{TilesFixedStructure} &
\Default{false} &
Indicates that each active picture parameter set has the same values of the syntax elements related to tiles.
\\
\Option{MotionVectorsOverPicBoundaries} &
\Default{false} &
Indicates that no samples outside the picture boundaries are used for inter prediction.
\\
\Option{MaxBytesPerPicDenom} &
\Default{2} &
Indicates a number of bytes not exceeded by the sum of the sizes of the VCL NAL units associated with any coded picture.
\\
\Option{MaxBitsPerMinCuDenom} &
\Default{1} &
Indicates an upper bound for the number of bits of coding_unit() data.
\\
\Option{Log2MaxMvLengthHorizontal} &
\Default{15} &
Indicate the maximum absolute value of a decoded horizontal MV component in quarter-pel luma units.
\\
\Option{Log2MaxMvLengthVertical} &
\Default{15} &
Indicate the maximum absolute value of a decoded vertical MV component in quarter-pel luma units.
\\
\end{OptionTableNoShorthand}


\begin{OptionTableNoShorthand}{Range Extensions (Version 2) tool parameters}{tab:rext-tools}

\Option{CostMode} &
\Default{lossy} &
Specifies the cost mode to use.
\par
\begin{tabular}{lp{0.3\textwidth}}
  lossy                   & $cost=distortion+\lambda \times bits$ \\
  sequence_level_lossless & $cost=distortion / \lambda + bits$. \\
  lossless                & As with sequence_level_lossless, but QP is also set to 0 (this will be deprecated in the future) \\
  mixed_lossless_lossy    & As with sequence_level_lossless, but QP'=4 is used for pre-estimates of transquant-bypass blocks \\
\end{tabular}
\\

\Option{ExtendedPrecision} &
\Default{false} &
Specifies the use of extended_precision_processing flag. Note that unless the HIGH_BIT_DEPTH_SUPPORT macro in TypeDef.h is enabled, all internal bit depths must be 8 when the ExtendedPrecision setting is enabled.
This setting is only valid for the 16-bit RExt profiles.
\\

\Option{HighPrecisionPredictionWeighting} &
\Default{false} &
Specifies the value of high_precision_prediction_weighting_flag. This setting is only valid for the 16-bit or 4:4:4 RExt profiles.
\\

\Option{CrossComponentPrediction} &
\Default{false} &
When true, specifies the use of the cross component prediction tool (4:4:4 processing only). Version 1 and some Version 2 (RExt) profiles require this to be false.
\\

\Option{ReconBasedCrossCPredictionEstimate} &
\Default{false} &
If true, then when determining the alpha value for cross-component prediction, use the reconstructed residual rather than the pre-transform encoder-side residual
\\

\Option{SaoLumaOffsetBitShift}
\Option{SaoChromaOffsetBitShift}&
\Default{0}
\Default{0} &
Specifies the shift to apply to the SAO parameters. If negative, an estimate will be calculated based upon the initial QP. Version 1 and some Version 2 (RExt) profiles require this to be 0.
\\

\Option{TransformSkipLog2MaxSize} &
\Default{2} &
Specifies the maximum TU size for which transform-skip can be used; the minimum value is 2. Version 1 and some Version 2 (RExt) profiles require this to be 2.
\\

\Option{ImplicitResidualDPCM} &
\Default{false} &
When true, specifies the use of the implicitly signalled residual RDPCM tool (for intra). Version 1 and some Version 2 (RExt) profiles require this to be false.
\\

\Option{ExplicitResidualDPCM} &
\Default{false} &
When true, specifies the use of the explicitly signalled residual RDPCM tool (for intra-block-copy and inter). Version 1 and some Version 2 (RExt) profiles require this to be false.
\\

\Option{ResidualRotation} &
\Default{false} &
When true, specifies the use of the residual rotation tool. Version 1 and some Version 2 (RExt) profiles require this to be false.
\\

\Option{SingleSignificanceMapContext} &
\Default{false} &
When true, specifies the use of a single significance map context for transform-skipped and transquant-bypassed TUs. Version 1 and some Version 2 (RExt) profiles require this to be false.
\\

\Option{GolombRiceParameterAdaptation} &
\Default{false} &
When true, enable the adaptation of the Golomb-Rice parameter over the course of each slice. Version 1 and some Version 2 (RExt) profiles require this to be false.
\\

\Option{AlignCABACBeforeBypass} &
\Default{false} &
When true, align the CABAC engine to a defined fraction of a bit prior to coding bypass data (including sign bits) when coeff_abs_level_remaining syntax elements are present in the group.
This must always be true for the high-throughput-RExt profile, and false otherwise.
\\

\Option{IntraReferenceSmoothing} &
\Default{true} &
When true, enable intra reference smoothing, otherwise disable it. Version 1 and some Version 2 (RExt) profiles require this to be true.
\\

\end{OptionTableNoShorthand}

\subsection{Encoder SEI parameters}
The table below lists the SEI messages defined for Version 1 and Range-Extensions, and if available, the respective table that lists the controls within the HM Encoder to include the messages within the bit stream.

\begin{SEIListTable}{List of Version 1 and RExt SEI messages}
    0 & Buffering period                         & Table \ref{tab:sei-buffering-period} \\
    1 & Picture timing                           & Table \ref{tab:sei-picture-timing} \\
    2 & Pan-scan rectangle                       & (Not handled)\\
    3 & Filler payload                           & (Not handled)\\
    4 & User data registered by Rec. ITU-T T.35  & (Not handled)\\
    5 & User data unregistered                   & Decoded only\\
    6 & Recovery point                           & Table \ref{tab:sei-recovery-point} \\
    9 & Scene information                        & (Not handled)\\
   15 & Picture snapshot                         & (Not handled)\\
   16 & Progressive refinement segment start     & (Not handled)\\
   17 & Progressive refinement segment end       & (Not handled)\\
   19 & Film grain characteristics               & (Not handled)\\
   22 & Post-filter hint                         & (Not handled)\\
   23 & Tone mapping information                 & Table \ref{tab:sei-tone-mapping-info} \\
   45 & Frame packing arrangement                & Table \ref{tab:sei-frame-packing-arrangement} \\
   47 & Display orientation                      & Table \ref{tab:sei-display-orientation} \\
  128 & Structure of pictures information        & Table \ref{tab:sei-sop-info} \\
  129 & Active parameter sets                    & Table \ref{tab:sei-active-parameter-sets} \\
  130 & Decoding unit information                & Table \ref{tab:sei-decoding-unit-info} \\
  131 & Temporal sub-layer zero index            & Table \ref{tab:sei-temporal-level-0} \\
  132 & Decoded picture hash                     & Table \ref{tab:sei-decoded-picture-hash} \\
  133 & Scalable nesting                         & Table \ref{tab:sei-scalable-nesting} \\
  134 & Region refresh information               & Table \ref{tab:sei-region-refresh-info} \\
  135 & No display                               & Table \ref{tab:sei-no-display} \\
  136 & Time code                                & Table \ref{tab:sei-time-code} \\
  137 & Mastering display colour volume          & Table \ref{tab:sei-mastering-display-colour-volume} \\
  138 & Segmented rectangular frame packing arrangement & Table \ref{tab:sei-seg-rect-fpa}\\
  139 & Temporal motion-constrained tile sets    & Table \ref{tab:sei-tmcts} \\
  140 & Chroma resampling filter hint            & Table \ref{tab:chroma-resampling-filter-hint} \\
  141 & Knee function information                & Table \ref{tab:sei-knee-function} \\
  142 & Colour remapping information             & (Not handled)\\
  143 & Deinterlaced field identification        & (Not handled)\\
\end{SEIListTable}
%%
%% SEI messages
%%

  

\begin{OptionTableNoShorthand}{Buffering period SEI message encoder parameters}{tab:sei-buffering-period}
\Option{SEIBufferingPeriod} &
\Default{0} &
Enables or disables the insertion of the Buffering period
SEI messages. This option has no effect if VuiParametersPresent is disabled.
SEIBufferingPeriod requires SEIActiveParameterSets to be enabled.
\\
\end{OptionTableNoShorthand}



\begin{OptionTableNoShorthand}{Picture timing SEI message encoder parameters}{tab:sei-picture-timing}
\Option{SEIPictureTiming} &
\Default{0} &
Enables or disables the insertion of the Picture timing
SEI messages. This option has no effect if VuiParametersPresent is disabled.
\\
\end{OptionTableNoShorthand}



\begin{OptionTableNoShorthand}{Recovery point SEI message encoder parameters}{tab:sei-recovery-point}
\Option{SEIRecoveryPoint} &
\Default{0} &
Enables or disables the insertion of the Recovery point
SEI messages.
\\
\end{OptionTableNoShorthand}



\begin{OptionTableNoShorthand}{Tone mapping information SEI message encoder parameters}{tab:sei-tone-mapping-info}
\Option{SEIToneMappingInfo} &
\Default{0} &
Enables or disables the insertion of the Tone Mapping SEI message.
\\
\Option{SEIToneMapId} &
\Default{0} &
Specifies Id of Tone Mapping SEI message for a given session.
\\
\Option{SEIToneMapCancelFlag} &
\Default{false} &
Indicates that Tone Mapping SEI message cancels the persistance or follows.
\\
\Option{SEIToneMapPersistenceFlag} &
\Default{true} &
Specifies the persistence of the Tone Mapping SEI message.
\\
\Option{SEIToneMapCodedDataBitDepth} &
\Default{8} &
Specifies Coded Data BitDepth of Tone Mapping SEI messages.
\\
\Option{SEIToneMapTargetBitDepth} &
\Default{8} &
Specifies Output BitDepth of Tome mapping function.
\\
\Option{SEIToneMapModelId} &
\Default{0} &
Specifies Model utilized for mapping coded data into 
target_bit_depth range.
\par
\begin{tabular}{cp{0.35\textwidth}}
  0 & linear mapping with clipping \\
  1 & sigmoidal mapping \\
  2 & user-defined table mapping \\
  3 & piece-wise linear mapping \\
  4 & luminance dynamic range mapping \\
\end{tabular}
\\
\Option{SEIToneMapMinValue} &
\Default{0} &
Specifies the minimum value in mode 0.
\\
\Option{SEIToneMapMaxValue} &
\Default{1023} &
Specifies the maxmum value in mode 0.
\\
\Option{SEIToneMapSigmoidMidpoint} &
\Default{512} &
Specifies the centre point in mode 1.
\\
\Option{SEIToneMapSigmoidWidth} &
\Default{960} &
Specifies the distance between 5% and 95% values of 
the target_bit_depth in mode 1.
\\
\Option{SEIToneMapStartOfCodedInterval} &
\Default{\None} &
Array of user-defined mapping table. 
Default table can be set to the following:
\par
\begin{tabular}{cp{0.35\textwidth}}
0 12 24 36 48 60 72 84 96 108 120 132 144 156 168 180 
\\
192 192 196 204 208 216 220 228 232 240 248 252 260 264 
\\
272 276 284 292 292 296 300 304 308 312 320 324 328 332 
\\
336 344 348 352 356 360 368 372 376 380 384 388 396 400 
\\
404 408 412 420 424 428 432 436 444 444 444 448 452 456 
\\
460 464 468 472 476 476 480 484 488 492 496 500 504 508 
\\
508 512 516 520 524 528 532 536 540 540 544 548 552 556 
\\
560 564 568 572 572 576 580 584 588 592 596 600 604 604 
\\
608 612 616 620 624 628 632 636 636 640 644 648 652 656 
\\
660 664 668 672 672 672 676 680 680 684 688 692 692 696 
\\
700 704 704 708 712 716 716 720 724 724 728 732 736 736 
\\
740 744 748 748 752 756 760 760 764 768 768 772 776 780 
\\
780 784 788 792 792 796 800 804 804 808 812 812 816 820 
\\
824 824 828 832 836 836 840 844 848 848 852 856 860 860 
\\
860 864 864 868 872 872 876 880 880 884 884 888 892 892 
\\
896 900 900 904 908 908 912 912 916 920 920 924 928 928 
\\
932 936 936 940 940 944 948 948 952 956 956 960 964 964 
\\
968 968 972 976 976 980 984 984 988 992 992 996 996 1000 
\\
1004 1004 1008 1012 1012 1016 1020 1024
\\
\end{tabular}
\\
\Option{SEIToneMapNumPivots} &
\Default{0} &
Specifies the number of pivot points in mode 3.
\\
\Option{SEIToneMapCodedPivotValue} &
\Default{\None} &
Array of coded pivot point in mode 3.
A suggested table is:
\par
\begin{tabular}{cp{0.45\textwidth}}
64 128 256 512 768
\end{tabular}
\\
\Option{SEIToneMapTargetPivotValue} &
\Default{\None} &
Array of target pivot point in mode 3.
A suggested table is:
\par
\begin{tabular}{cp{0.45\textwidth}}
48 73 111 168 215
\end{tabular}
\\
\Option{SEIToneMap...} \Option{CameraIsoSpeedIdc} &
\Default{0} &
Indicates the camera ISO speed for daylight illumination.
\\
\Option{SEIToneMap...} \Option{CameraIsoSpeedValue} &
\Default{400} &
Specifies the camera ISO speed for daylight illumination of Extended_ISO.
\\
\Option{SEIToneMap...} \Option{ExposureIndexIdc} &
\Default{0} &
Indicates the exposure index setting of the camera.
\\
\Option{SEIToneMap...} \Option{ExposureIndexValue} &
\Default{400} &
Specifies the exposure index setting of the cameran of Extended_ISO.
\\
\Option{SEIToneMapExposure...} \Option{CompensationValueSignFlag} &
\Default{0} &
Specifies the sign of ExposureCompensationValue.
\\
\Option{SEIToneMapExposure...} \Option{CompensationValueNumerator} &
\Default{0} &
Specifies the numerator of ExposureCompensationValue.
\\
\Option{SEIToneMapExposure...} \Option{CompensationValueDenomIdc} &
\Default{2} &
Specifies the denominator of ExposureCompensationValue.
\\
\Option{SEIToneMapRef...} \Option{ScreenLuminanceWhite} &
\Default{350} &
Specifies reference screen brightness setting in units of candela per square metre.
\\
\Option{SEIToneMapExtended...} \Option{RangeWhiteLevel} &
\Default{800} &
Indicates the luminance dynamic range.
\\
\Option{SEIToneMapNominal...} \Option{BlackLevelLumaCodeValue} &
\Default{16} &
Specifies luma sample value of the nominal black level assigned decoded pictures.
\\
\Option{SEIToneMapNominal...} \Option{WhiteLevelLumaCodeValue} &
\Default{235} &
Specifies luma sample value of the nominal white level assigned decoded pictures.
\\
\Option{SEIToneMapExtended...} \Option{WhiteLevelLumaCodeValue} &
\Default{300} &
Specifies luma sample value of the extended dynamic range assigned decoded pictures.
\\
\end{OptionTableNoShorthand}



\begin{OptionTableNoShorthand}{Frame packing arrangement SEI message encoder parameters}{tab:sei-frame-packing-arrangement}
\Option{SEIFramePacking} &
\Default{0} &
Enables or disables the insertion of the Frame packing arrangement SEI messages.
\\
\Option{SEIFramePackingType} &
\Default{0} &
Indicates the arrangement type in the Frame packing arrangement SEI message.
This option has no effect if SEIFramePacking is disabled.
\par
\begin{tabular}{cp{0.35\textwidth}}
  3 & Side by Side \\
  4 & Top Bottom \\
  5 & Frame Alternate \\
\end{tabular}
\\
\Option{SEIFramePackingInterpretation} &
\Default{0} &
Indicates the constituent frames relationship in the Frame packing arrangement SEI message. 
This option has no effect if SEIFramePacking is disabled.
\par
\begin{tabular}{cp{0.35\textwidth}}
  0 & Unspecified \\
  1 & Frame 0 is associated with the left view of a stereo pair \\
  2 & Frame 0 is associated with the right view of a stereo pair \\
\end{tabular}
\\
\Option{SEIFramePackingQuincunx} &
\Default{0} &
Enables or disables the quincunx_sampling signalling in the
Frame packing arrangement SEI messages. This option has no
effect if SEIFramePacking is disabled.
\\
\Option{SEIFramePackingId} &
\Default{0} &
Indicates the session number in the Frame packing arrangement
SEI messages. This option has no effect if SEIFramePacking is
disabled.
\\
\end{OptionTableNoShorthand}



\begin{OptionTableNoShorthand}{Display orientation SEI message encoder parameters}{tab:sei-display-orientation}
\Option{SEIDisplayOrientation} &
\Default{0} &
Enables or disables the insertion of the Display orientation
SEI messages.
\par
\begin{tabular}{cp{0.20\textwidth}}
  0 & Disabled \\
  N: $0 < N < (2^{16} - 1)$ & Enable display orientation SEI message with 
  \mbox{anticlockwise_rotation = N} 
  and \mbox{display_orientation_repetition_period = 1} \\
\end{tabular}
\\
\end{OptionTableNoShorthand}



\begin{OptionTableNoShorthand}{Structure of pictures information SEI message encoder parameters}{tab:sei-sop-info}
\Option{SEISOPDescription} &
\Default{0} &
Enables or disables the insertion of the Structure of pictures information SEI messages.
\\
\end{OptionTableNoShorthand}



\begin{OptionTableNoShorthand}{Active parameter sets SEI message encoder parameters}{tab:sei-active-parameter-sets}
\Option{SEIActiveParameterSets} &
\Default{0} &
Enables or disables the insertion of the Active parameter sets
SEI messages.
\\
\end{OptionTableNoShorthand}



\begin{OptionTableNoShorthand}{Decoding unit information SEI message encoder parameters}{tab:sei-decoding-unit-info}
\Option{SEIDecodingUnitInfo} &
\Default{0} &
Enables or disables the insertion of the Decoding unit information
SEI messages. This option has no effect if VuiParametersPresent is disabled.
\\
\end{OptionTableNoShorthand}



\begin{OptionTableNoShorthand}{Temporal sub-layer zero index SEI message encoder parameters}{tab:sei-temporal-level-0}
\Option{SEITemporalLevel0Index} &
\Default{0} &
Enables or disables the insertion of the Temporal level zero index
SEI messages.
\\
\end{OptionTableNoShorthand}



\begin{OptionTableNoShorthand}{Decoded picture hash SEI message encoder parameters}{tab:sei-decoded-picture-hash}
\Option{SEIDecodedPictureHash} &
\Default{0} &
Enables or disables the calculation and insertion of the Decoded picture hash
SEI messages.
\par
\begin{tabular}{cp{0.35\textwidth}}
  0 & Disabled \\
  1 & Transmits MD5 in SEI message and writes the value to the encoder
      log \\
  2 & Transmits CRC in SEI message and writes the value to the encoder
      log \\
  3 & Transmits checksum in SEI message and writes the value to the encoder
      log \\
\end{tabular}
\\
\end{OptionTableNoShorthand}



\begin{OptionTableNoShorthand}{Scalable nesting SEI message encoder parameters}{tab:sei-scalable-nesting}
\Option{SEIScalableNesting} &
\Default{0} &
Enables or disables the use of the scalable nesting SEI messages.
\\
\end{OptionTableNoShorthand}


        
\begin{OptionTableNoShorthand}{Region refresh information SEI message encoder parameters}{tab:sei-region-refresh-info}
\Option{SEIGradualDecodingRefreshInfo} &
\Default{0} &
Enables or disables the insertion of the Gradual decoding refresh information
SEI messages.
\\
\end{OptionTableNoShorthand}



\begin{OptionTableNoShorthand}{No display SEI message encoder parameters}{tab:sei-no-display}
\Option{SEINoDisplay} &
\Default{0} &
When non-zero, generate no-display SEI message for temporal layer N or higher.
\\
\end{OptionTableNoShorthand}


        
\begin{OptionTableNoShorthand}{Time code SEI message encoder parameters}{tab:sei-time-code}
\Option{SEITimeCodeEnabled} &
\Default{false} &
When true (non-zero), generate Time code SEI messages.
\\
\Option{SEITimeCodeNumClockTs} &
\Default{0} &
Number of clock time sets, in the range of 0 to 3 (inclusive).
\\
\Option{SEITimeCodeTimeStampFlag} &
\Default{\None} &
Time stamp flag associated to each time set (comma or space separated list of entries).
\\
\Option{SEITimeCodeFieldBasedFlag} &
\Default{\None} &
Field based flag associated to each time set (comma or space separated list of entries).
\\
\Option{SEITimeCodeCountingType} &
\Default{\None} &
Counting type associated to each time set (comma or space separated list of entries).
\\
\Option{SEITimeCodeFullTsFlag} &
\Default{\None} &
Full time stamp flag associated to each time set (comma or space separated list of entries).
\\
\Option{SEITimeCodeDiscontinuityFlag} &
\Default{\None} &
Discontinuity flag associated to each time set (comma or space separated list of entries).
\\
\Option{SEITimeCodeCntDroppedFlag} &
\Default{\None} &
Counter dropped flag associated to each time set (comma or space separated list of entries).
\\
\Option{SEITimeCodeNumFrames} &
\Default{\None} &
Number of frames associated to each time set (comma or space separated list of entries).
\\
\Option{SEITimeCodeSecondsFlag} &
\Default{\None} &
Flag to signal seconds value presence in each time set (comma or space separated list of entries).
\\
\Option{SEITimeCodeMinutesFlag} &
\Default{\None} &
Flag to signal minutes value presence in each time set (comma or space separated list of entries).
\\
\Option{SEITimeCodeHoursFlag} &
\Default{\None} &
Flag to signal hours value presence in each time set (comma or space separated list of entries).
\\
\Option{SEITimeCodeSecondsValue} &
\Default{\None} &
Seconds value for each time set (comma or space separated list of entries).
\\
\Option{SEITimeCodeMinutesValue} &
\Default{\None} &
Minutes value for each time set (comma or space separated list of entries).
\\
\Option{SEITimeCodeHoursValue} &
\Default{\None} &
Hours value for each time set (comma or space separated list of entries).
\\
\Option{SEITimeCodeOffsetLength} &
\Default{\None} &
Time offset length associated to each time set (comma or space separated list of entries).
\\
\Option{SEITimeCodeTimeOffset} &
\Default{\None} &
Time offset associated to each time set (comma or space separated list of entries).
\\
\end{OptionTableNoShorthand}



\begin{OptionTableNoShorthand}{Mastering display colour volume SEI message encoder parameters}{tab:sei-mastering-display-colour-volume}
\Option{SEIMasteringDisplayColourVolume} &
\Default{false} &
When true (non-zero), generate Mastering display colour volume SEI message.
\\
\Option{SEIMasteringDisplayMaxLuminance} &
\Default{10000} &
Specifies the mastering display maximum luminance value in units of 1/10000 candela per square metre.
\\
\Option{SEIMasteringDisplayMinLuminance} &
\Default{0} &
Specifies the mastering display minimum luminance value in units of 1/10000 candela per square metre.
\\
\Option{SEIMasteringDisplayPrimaries} &
\Default{0,50000, 0,0, 50000,0} &
Mastering display primaries for all three colour planes in CIE xy coordinates in increments of 1/50000 (results in the ranges 0 to 50000 inclusive).
\\
\Option{SEIMasteringDisplayWhitePoint} &
\Default{16667, 16667} &
Mastering display white point CIE xy coordinates in normalized increments of 1/50000 (e.g. 0.333 = 16667).
\\
\end{OptionTableNoShorthand}



\begin{OptionTableNoShorthand}{Segmented rectangular frame packing arrangement SEI message encoder parameters}{tab:sei-seg-rect-fpa}
\Option{SEISegmentedRectFramePacking} &
\Default{0} &
Controls generation of segmented rectangular frame packing SEI messages.
\\
\Option{SEISegmentedRectFramePackingCancel} &
\Default{false} &
If true, cancels the persistence of any previous SRFPA SEI message.
\\
\Option{SEISegmentedRectFramePackingType} &
\Default{0} &
Specifies the arrangement of the frames in the reconstructed picture.
\\
\Option{SEISegmentedRectFramePackingPersistence} &
\Default{false} &
If false the SEI applies to the current frame only.
\\
\end{OptionTableNoShorthand}



\begin{OptionTableNoShorthand}{Temporal motion-constrained tile sets SEI message encoder parameters}{tab:sei-tmcts}
\Option{SEITempMotionConstrainedTileSets} &
\Default{false} &
When true (non-zero), generates example temporal motion constrained tile sets SEI messages.
\\
\end{OptionTableNoShorthand}



\begin{OptionTableNoShorthand}{Chroma resampling filter hint SEI message encoder parameters}{tab:chroma-resampling-filter-hint}
\Option{SEIChromaSamplingFilterHint} &
\Default{false} &
When true (non-zero), generates example chroma sampling filter hint SEI messages.
\\
\Option{SEIChromaSamplingHorizontalFilterType} &
\Default{2} &
Defines the index of the chroma sampling horizontal filter:
\par
\begin{tabular}{cp{0.35\textwidth}}
  0 & Unspecified \\
  1 & Filters signalled within the SEI message \\
  2 & Filters as described by SMPTE RP 2050-1:2012\\
\end{tabular}
\\
\Option{SEIChromaSamplingVerticalFilterType} &
\Default{2} &
Defines the index of the chroma sampling vertical filter:
\par
\begin{tabular}{cp{0.35\textwidth}}
  0 & Unspecified \\
  1 & Filters signalled within the SEI message \\
  2 & Filters as described in the 5/3 filter description of ITU-T Rec. T.800 | ISO/IEC 15444-1\\
\end{tabular}
\\
\end{OptionTableNoShorthand}



\begin{OptionTableNoShorthand}{Knee function SEI message encoder parameters}{tab:sei-knee-function}
\Option{SEIKneeFunctionInfo} &
\Default{false} &
Enables (true) or disables (false) the insertion of the Knee function SEI messages.
\\
\Option{SEIKneeFunctionId} &
\Default{0} &
Specifies Id of Knee function SEI message for a given session.
\\
\Option{SEIKneeFunctionCancelFlag} &
\Default{false} &
Indicates that Knee function SEI message cancels the persistance (true) or follows (false).
\\
\Option{SEIKneeFunctionPersistenceFlag} &
\Default{true} &
Specifies the persistence of the Knee function SEI message.
\\
\Option{SEIKneeFunctionInputDrange} &
\Default{1000} &
Specifies the peak luminance level for the input picture of Knee function SEI messages.
\\
\Option{SEIKneeFunctionInputDispLuminance} &
\Default{100} &
Specifies the expected display brightness for the input picture of Knee function SEI messages.
\\
\Option{SEIKneeFunctionOutputDrange} &
\Default{4000} &
Specifies the peak luminance level for the output picture of Knee function SEI messages.
\\
\Option{SEIKneeFunctionOutputDispLuminance} &
\Default{800} &
Specifies the expected display brightness for the output picture of Knee function SEI messages.
\\
\Option{SEIKneeFunctionNumKneePointsMinus1} &
\Default{2} &
Specifies the number of knee points - 1.
\\
\Option{SEIKneeFunctionInputKneePointValue} &
\Default{} &
Array of input knee point. Default table can be set to the following:
\par
\begin{tabular}{cp{0.45\textwidth}}
600 800 900
\end{tabular}
\\
\Option{SEIKneeFunctionOutputKneePointValue} &
\Default{} &
Array of output knee point. Default table can be set to the following:
\par
\begin{tabular}{cp{0.45\textwidth}}
100 250 450
\end{tabular}
\\
\end{OptionTableNoShorthand}



%\Option{SEITimeCode} &
%\Default{false} &
%When true, generate time code SEI messages.
%\\

%%
%%
%%
\subsection{Hardcoded encoder parameters}
\begin{MacroTable}{CommonDef.h constants}
ADAPT_SR_SCALE &
1 &
Defines a scaling factor used to derive the motion search range is
adaptive (see ASR configuration parameter). Default value is 1.
\\

MAX_GOP &
64 &
maximum size of value of hierarchical GOP.
\\

MAX_NUM_REF &
4 &
maximum number of multiple reference frames
\\

MAX_NUM_REF_LC &
8 &
maximum number of combined reference frames
\\

AMVP_MAX_NUM_CANDS &
2 &
maximum number of final candidates
\\

AMVP_MAX_NUM_CANDS_MEM &
3 &
\\

MRG_MAX_NUM_CANDS &
5 &
\\

DYN_REF_FREE &
off &
dynamic free of reference memories
\\

MAX_TLAYER &
8 &
maximum number of temporal layers
\\

ADAPT_SR_SCALE &
on &
division factor for adaptive search range
\\

EARLY_SKIP_THRES &
1.5 &
early skip if RD < EARLY_SKIP_THRES*avg[BestSkipRD]
\\

MAX_NUM_REF_PICS &
16 &
\\

MAX_CHROMA_FORMAT_IDC &
3 &
\\
\end{MacroTable}

\subsubsection*{TypeDef.h}
Numerous constants that guard individual adoptions are defined within
\url{source/Lib/TLibCommon/TypeDef.h}.


%%
%%
%%
\clearpage
\section{Using the decoder}
\subsection{General}
\begin{verbatim}
TAppDecoder -b str.bin -o dec.yuv [options]
\end{verbatim}

\begin{OptionTableNoShorthand}{Decoder options}{tab:decoder-options}
\Option{(--help)} &
%\ShortOption{\None} &
\Default{\None} &
Prints usage information.
\\

\Option{BitStreamFile (-b)} &
%\ShortOption{-b} &
\Default{\NotSet} &
Defines the input bit stream file name.
\\

\Option{ReconFile (-o)} &
%\ShortOption{-o} &
\Default{\NotSet} &
Defines reconstructed YUV file name. If empty, no file is generated.
\\

\Option{SkipFrames (-s)} &
%\ShortOption{-s} &
\Default{0} &
Defines the number of pictures in decoding order to skip.
\\

\Option{MaxTemporalLayer (-t)} &
%\ShortOption{-t} &
\Default{-1} &
Defines the maximum temporal layer to be decoded. If -1, then all layers are decoded.
\\

\Option{TarDecLayerIdSetFile (-l)} &
%\ShortOption{-t} &
\Default{\NotSet} &
Specifies the targetDecLayerIdSet file name. The file would contain white-space separated LayerId values of the layers that are to be decoded.
Omitting the parameter, or using a value of -1 in the file decodes all layers.
\\

\Option{OutputBitDepth (-d)} &
%\ShortOption{-d} &
\Default{0 \\ (Native)} &
Specifies the luma bit-depth of the reconstructed YUV file (the value 0 indicates
that the native bit-depth is used)
\\

\Option{OutputBitDepthC} &
%\ShortOption{\None} &
\Default{0 \\ (Native)} &
Defines the chroma bit-depth of the reconstructed YUV file (the value 0 indicates
that the native bit-depth is used)
\\

\Option{SEIDecodedPictureHash} &
%\ShortOption{\None} &
\Default{1} &
Enable or disable verification of any Picture hash SEI messages. When
this parameter is set to 0, the feature is disabled and all messages are
ignored. When set to 1 (default), the feature is enabled and the decoder
has the following behaviour:
\begin{itemize}
\item
  If Picture hash SEI messages are included in the bit stream, the same type
  of hash is calculated for each decoded picture and written to the
  log together with an indication whether the calculted value matches
  the value in the SEI message. 
  Decoding will continue even if there is a mismatch.

\item
  After decoding is complete, if any MD5sum comparison failed, a warning
  is printed and the decoder exits with the status EXIT_FAILURE

\item
  The per-picture MD5 log message has the following formats:
  [MD5:d41d8cd98f00b204e9800998ecf8427e,(OK)],
  [MD5:d41d8cd98f00b204e9800998ecf8427e,(unk)],
  [MD5:d41d8cd98f00b204e9800998ecf8427e,(***ERROR***)] [rxMD5:b9e1...]
  where, "(unk)" implies that no MD5 was signalled for this picture,
  "(OK)" implies that the decoder agrees with the signalled MD5,
  "(***ERROR***)" implies that the decoder disagrees with the signalled
  MD5. "[rxMD5:...]" is the signalled MD5 if different.
\end{itemize}
\\

\Option{OutputDecodedSEIMessagesFilename} &
%\ShortOption{\None} &
\Default{\NotSet} &
When a non-empty file name is specified, information regarding any decoded SEI messages will be output to the indicated file. If the file name is '-', then stdout is used instead.
\\

\Option{RespectDefDispWindow (-w)} &
%\ShortOption{-w} &
\Default{0} &
Video region to be output by the decoder.
\par
\begin{tabular}{cp{0.45\textwidth}}
  0 & Output content inside the conformance window. \\
  1 & Output content inside the default window. \\
\end{tabular}
\\

\Option{OutputColourSpaceConvert} &
\Default{\NotSet} &
Specifies the colour space conversion to apply to 444 video. Permitted values are:
\par
\begin{tabular}{lp{0.45\textwidth}}
  UNCHANGED   & No colour space conversion is applied \\
  YCrCbToYCbCr & Swap the second and third components \\
  GBRtoRGB     & Reorder the three components \\
\end{tabular}
If no value is specified, no colour space conversion is applied. The list may eventually also include RGB to YCbCr or YCgCo conversions.\\
\\

\Option{SEINoDisplay} &
\Default{false} &
When true, do not output frames for which there is an SEI NoDisplay message.
\\

\Option{ClipOutputVideoToRec709Range} &
%\ShortOption{\None} &
\Default{0} &
If 1 then clip output video to the Rec. 709 Range on saving when OutputBitDepth is less than InternalBitDepth.
\\

\end{OptionTableNoShorthand}


\subsection{Using the decoder analyser}
If the decoder is compiled with the macro RExt__DECODER_DEBUG_BIT_STATISTICS defined as 1 (either externally, or by editing TypeDef.h), the decoder will gather fractional bit counts associated with the different syntax elements, producing a table of the number of bits per syntax element, and where appropriate, according to block size and colour component/channel.
The Linux makefile will compile both the analyser and standard version when the `all' or `everything' target is used (where the latter will also build  high-bit-depth executables).


\end{document}
